%REDES SOCIAIS-------------------------------------------------------

% REFERÊNCIAS DESTE CAPÍTULO----------------------------------
%
% SOCIAL MEDIA 69 TO 15
% DAS REDES SOCIAIS A INOVAÇÃO
% IMPACTO DAS REDES SOCIAIS JOVENS ACADEMICOS
% REDES SOCIAIS COMO MÉTODO PARA CI

\chapter{REDES SOCIAIS}
\label{chap:redesSociais}
Desde seu surgimento, \textit{sites} de redes sociais, como \textit{Facebook} e \textit{Twitter}, vêm atraindo milhões de usários ao redor do mundo. Segundo estatísticas do site Dossier Statista (REFERENCIAR DOSSIER STATISTA), em 2017 haviam 2.46 bilhões de usários de redes sociais ao redor do mundo e é estimado que em 2019 esse número suba para 2.77 bilhões.

Tentar definir em termos precisos o que seria uma rede social se prova difícil devido a grande variedade de serviços independentes e integrados de serviços de comunicação. Simplesmente definir como serviços que permitem aproximar pessoas de forma digital se torna uma definição muito ampla.

\section{CONCEITOS FUNDAMENTAIS}
\label{sec:conceitosRedesSociais}
 Revisões de literatura permitiram extrair 4 características principais recorrentes nos serviços de \textit{networking} social.

(REFERENCIAR SOCIAL MEDIA DEFINITION PEGAR ARTIGO NO SCIENCE DIRECT: https://www.sciencedirect.com/science/article/abs/pii/S0308596115001172)

\begin{enumerate}
	\item social networking services are interactive Web 2.0 Internet-based applications,
	\item user-generated content (UGC), such as user-submitted digital photos, text posts, "tagging", online comments, and diary-style "web logs" (blogs), is the lifeblood of the SNS organism,
	\item users create service-specific profiles for the site or app that are designed and maintained by the SNS organization, and
	\item social networking services facilitate the development of social networks online by connecting a user's profile with those of other individuals or groups.
\end{enumerate}

Breve histórico de algumas redes sociais famosas.

Tipos de redes sociais brevemente explicados.

\begin{enumerate}
	\item \textit{Networking}
	\item \textit{Blogging}
	\item \textit{Microblogging}
	\item \textit{Photo Sharing}
	\item \textit{Video Sharing}
\end{enumerate}

Usos diferentes de redes sociais

\begin{enumerate}
	\item \textit{Real time}
	\item \textit{Location based}
	\item Mercados de nicho
	\item Ciência
	\item Educação
	\begin{enumerate}
		\item Profissional
		\item Currículo
		\item Aprendizado
	\end{enumerate}
	\item Procuro de emprego
	\item Serviços de \textit{hosting}
	\item Trocas
\end{enumerate}

\section{IMPACTOS E RELEVÂNCIA}
\label{sec:classRedesSociais}
Alguns benefícios

Alguns problemas

Relevância dos estudos na área

\section{\textit{FACEBOOK}}
\label{sec:facebookRedesSociais}
Breve histórico \textit{Facebook}.

\textit{Facebook} utilizado em negócios para empresas.

Explicação das métricas de avaliação de desempenho de postagem pelo \textit{Facebook}. Se encontram no quadro da metodologia