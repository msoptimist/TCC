% INTRODUÇÃO-------------------------------------------------------------------

% REFERÊNCIAS DESTE CAPÍTULO
%
%	PREDICTING SOCIAL MEDIA PERFORMANCE
%	DOSSIER STATISTA
%	https://www.internetworldstats.com/stats.htm

\chapter{INTRODUÇÃO}
\label{chap:introducao}

Segundo o \textit{site} <https://www.internetworldstats.com/stats.htm>, 54\% da população mundial está conectada na \textit{internet}, somando um total de mais de 4 bilhões de pessoas \textit{online} no mundo todo. Desse total, segundo pesquisas da (DOSSIER STATISTA) 2.62 bilhões se conectam através de redes sociais e a previsão é de 3.02 até o fim de 2021.

Com a crescente presença de serviços digitais na vida cotidiana da população, como o uso de redes sociais, serviços de \textit{streaming}, lojas e compras \textit{online} há uma quantidade enorme de dados está sendo armazenada a todo momento em todos os lugares do mundo (CITAR DOSSIER STATISTA PARA NUMERO DE USUÁRIOS). Acumular dados não é suficiente por si só, tão importante quanto é extrair informações destes. Porém, analisar os dados de maneira manual requer interpretação por parte do analista, sendo muitas vezes ineficientes, demorados, custosos além de propensos à subjetividade do analista. (REFERENCIAR ESSA PARTE)

(REFERENCIA FAYYAD) diz que neste cenário há uma grande necessidade de novas teorias e ferramentas computacionais a fim de auxiliar na extração de conhecimento dos grandes e cada vez maiores volumes de dados. (FINALIZAR PENSAMENTO)

O processo de Descoberta de Conhecimento em Bases de Dados (\textit{Knowledge Discovery in Databases} - KDD), termo cunhado por (REFERENCIA FAYYAD) na realização do primeiro \textit{workshop} a respeito do tema, foi uma resposta aos problemas causados por esse grande acumulo de dados. (REFERENCIAR KDD)

O KDD é um processo iterativo não trivial dividido em 9 etapas dependentes de interações do usuário. O processo tem por objetivo identificar informações válidas, novas, potencialmente úteis e compreensíveis em um grupo de dados, segundo (REFERENCIA FAYYAD). Essas informações extraídas também precisam ser, até certo grau de certeza, válidas para um novo conjunto de dados.

O processo de KDD envolve desde a compreensão dos dados armazenados na base, sua seleção e pré-processamento, passando por etapas de transformação e aplicação de técnicas de mineração de dados, terminando nas etapas de identificação dos padrões gerados que se enquadram nos requisitos de conhecimento do processo.

Entre as etapas do processo de KDD, a mineração de dados é uma das que possui grande ênfase tanto nas áreas acadêmicas quanto nas áreas práticas devido as grandes quantidades de métodos e resultados disponíveis e testados. O processo de mineração de dados envolve a descoberta de padrões a partir de dados e a adaptação de modelos para melhor acomodar os dados existentes, utilizando-se de técnicas de muitas áreas diversas, como aprendizagem de máquina, reconhecimento de padrões e estatística. (REFERENCIAR ESSA PARTE)

Dentro dos diversos métodos de mineração de dados, sendo os mais comuns os de classificação, clusterização e regressão, existem uma extensa gama de algoritmos, porém todos com funcionamentos e fundamentos parecidos. (REFERENCIA FAYYAD) diz que os métodos de mineração de dados podem ser compostos de três algoritmos diferentes: modelo de representação, modelo de avaliação, e busca.
	
Como foi citado anteriormente, somente ter acesso a determinados dados não garante a completa compreensão do problema, porém, com a aplicação de processos como o KDD, pode-se tirar conclusões úteis a respeito dos dados tratados. Este trabalho tem como objetivo utilizar o processo do KDD numa base de dados retirada de uma rede social, a fim de encontrar um modelo de previsão para as métricas de avaliação estabelecidas pelo \textit{Facebook}.

\section{DESCRIÇÃO DO PROBLEMA E MOTIVAÇÃO}
\label{sec:descricaomotivacao}

O grande aumento do volume de dados acumulados por diversos tipos de aplicações pode gerar um grande problema, uma vez que nem sempre somente tê-los significa ter algum conhecimento a respeito do problema em questão. Erros podem ser cometidos sem uma devida análise rigorosa e metódica a fim de se extrair informações relevantes, levando a conclusões erradas e na base de achismos.

Aplicar o processo de KDD se torna cada vez mais interessante nesse cenário superpopuloso de dados uma vez que estabelece um processo que pode ser seguido e replicado em diversos tipos de situações diferentes. A base de dados utilizada no trabalho foi escolhida por possuir dados de uma situação real, além de serem relativamente recentes e com poucos trabalhos explorando os diversos tratamentos que os dados podem receber, permitindo assim uma grande gama de abordagens diferentes possíveis.

\section{OBJETIVOS}
\label{sec:objetivos}
Esta Seção apresenta o obejtivos geral e os objetivos específicos deste trabalho. Na Subseção \ref{subsec:objGerais} se encontra o objetivo geral e na Subseção \ref{subsec:objEspecificos} se encontram os objetivos específicos.

\subsection{Objetivos gerais}
\label{subsec:objGerais}
O objetivo geral deste trabalho é aplicar o processo de KDD numa base de dados de domínio público a fim de criar um modelo de previsão de sucesso de postagem.

\subsection{Objetivos específicos}
\label{subsec:objEspecificos}
Como objetivos específicos têm-se:
\begin{itemize}
	\item Compreender o funcionamento do processo de KDD;
	\item Analisar as etapas do KDD, identificado técnicas que podem ser aplicadas;
	\item Aplicar o processo de KDD na base de dados;
	\item Realizar experimentos e analisar os resultados obtidos por meio de comparação estatística com outros trabalhos da área.
\end{itemize}

\section{ORGANIZAÇÃO DO TRABALHO}
\label{sec:organizacaoTrabalho}
Este trabalho encontra-se dividido nos seguintes sete capítulos:

\begin{itemize}
	\item Capítulo \ref{chap:introducao}: Capítulo introdutório de contextualização do trabalho, apresentando em linhas gerais a situação atual, a motivação que levou a idealização deste trabalho e os objetivos a serem alcançados.

	\item Capítulo \ref{chap:descobertaConhecimento}: Este capítulo aborda os conceitos necessários para compreender o processo de KDD. Contém uma descrição das etapas que o constituem como um todo e entra mais a fundo em conceitos importantes da parte de mineração de dados.

	\item Capítulo \ref{chap:redesSociais}: O capítulo apresenta conceitos e definições a respeito de redes sociais, classificando-as e descrevendo sua relevância fora e dentro da área acadêmica. Também contém uma explicação a respeito de termos da rede social Facebook, sendo esta o local onde os dados utilizados neste trabalho foram retirados.

	\item Capítulo \ref{chap:revisaoSistematica}: Neste capítulo encontra-se uma explicação a respeito do método de revisão metodológica, ressaltando os pontos importantes do processo e a aplicação deste em bases de artigos acadêmicos, permitindo a identificação de trabalhos correlatos que serviram de referência para a elaboração deste.

	\item Capítulo \ref{chap:metodologia}: O capítulo descreve a base de dados utilizada para a realização dos experimentos deste trabalho, bem como as técnicas e métodos a serem aplicados na mesma, seguindo o processo definido no capítulo \ref{chap:descobertaConhecimento}.

	\item Capítulo \ref{chap:resultados}: Este capítulo contém a descrição e análise dos resultados obtidos após a aplicação dos métodos e técnicas descritos no capítulo \ref{chap:metodologia} sobre a base de dados deste trabalho.

	\item Capítulo \ref{chap:conclusao}: O capítulo contém possibilidades de continuações deste trabalho, bem como as considerações finais a respeito dos resultados obtidos e do cumprimento dos objetivos propostos.
\end{itemize}