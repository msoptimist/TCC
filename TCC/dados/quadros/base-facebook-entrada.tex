% BASE DE DADOS UTILIZADA NO TRABALHO---------------------------------------------------------------------

\begin{quadro}[!htb]
	\centering
	\caption{Descrição dos atributos de entrada da base de dados utilizada no trabalho.\label{qua:descBaseFacebookEntrada}}
	\small
	\begin{tabular}{|p{7cm}|p{7cm}|}
        		\hline
		\textbf{Atributo} & \textbf{Descrição} \\ \hline
		\textit{Page total likes} & Quantidade de \textit{likes} totais da página quando feita a postagem. \\ \hline
		\textit{Type} & Tipo de postagem, entre Fotos, \textit{Link}, Status e Vídeo. \\ \hline
		\textit{Category} & Categoria de tipo de propaganda utilizada internamente pela empresa, adicionado de forma manual na base de dados. \\ \hline
		\textit{Post Month} & Mês em que a postagem foi feita, retirado da data da postagem. \\ \hline
		\textit{Post Weekday} & Dia da semana em que a postagem foi feita, retirado da data da postagem. \\ \hline
		\textit{Post Hour} & Hora do dia em que a postagem foi feita, retirado da data da postagem. \\ \hline
		\textit{Paid} & Se o post usou os serviços de anúncios pagos do \textit{Facebook}. \\ \hline
    \end{tabular}
	\fonte{\cite{predictingSocialMedia}}
\end{quadro}