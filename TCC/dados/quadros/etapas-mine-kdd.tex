\begin{quadro}[!htb]
    \centering
    \caption{Etapas da mineração do KDD.\label{qua:etapasMineKDD}}
    \begin{tabular}{|p{7cm}|p{7cm}|}
        \hline
        \textbf{Etapas} & \textbf{Descrição} \\
        \hline
        5. Escolha da tarefa de mineração & Escolha da tarefa de mineração em sincronia com os objetivos levantados no passo 1 do processo: e.g. classificação, clusterização, regressão, etc. \\ \hline
        6. Escolha do algoritmo de mineração & Inclui tanto a seleção de métodos de busca por padrões quanto quais modelos e parâmetros são mais apropriados para os critérios do conhecimento a ser extraído. \\ \hline
        7. Mineração de dados & Geração dos padrões de interesse em determinada forma de representação. \\
        \hline
    \end{tabular}
    \fonte{REFERENCIAR DATA MINING A KNOWLEDGE DISCOVERY APPROACH}
\end{quadro}