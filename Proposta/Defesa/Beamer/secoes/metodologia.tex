\section{Metodologia}

\subsection{Etapas}
\begin{frame}
	\begin{enumerate}
		\item Levantamento bibliográfico;
		\item Compreensão do processo de KDD;
		\item Identificação de técnicas aplicáveis nas etapas do KDD;
		\item Aplicação do processo de KDD utilizando a ferramenta WEKA Knowledge Flow;
		\item Realização dos experimentos numa base de dados pública;
		\item Análise dos resultados.
	\end{enumerate}
\end{frame}

\subsection{Ferramentas}
\begin{frame}
	\begin{itemize}
		\item Aplicação de experimentos com apoio da WEKA (Waikato Enviroment for Knowledge Analysis);
		\item Ferramenta de auxílio desenvolvida na Nova Zelândia;
		\item WEKA Knowledge Flow permite a vizualização de processos como data-flow;
		\item 5 técnicas de regressão:
		\begin{enumerate}
			\item Regressão linear;
			\item K vizinhos mais próximos;
			\item Árvores de decisão;
			\item Máquinas de vetores de suporte;
			\item Perceptron multicamadas.
		\end{enumerate}
	\end{itemize}
\end{frame}

\subsection{Base de dados}
\begin{frame}
	\begin{itemize}
		\item Inicialmente descrita por Moro (2016) \cite{moro2016};
		\item A base contém 500 postagens publicadas durante o ano de 2014 na página do Facebook de uma marca de cosméticos;
		\item 19 atributos, sendo 7 referentes a postagem e 12 referentes a performance e impacto medidas pelo Facebook;
		\item Análise original utilizou máquinas de vetores de suporte do pacote rminer do software de análise estatística R;
		\item Melhores previsões de erros foram ao entorno de 27\%.
	\end{itemize}
\end{frame}