\section{Metodologia}

\subsection{Etapas}
\begin{frame}
	\begin{enumerate}
		\item Levantamento bibliográfico;
		\item Compreensão do processo de KDD;
		\item Identificação de técnicas aplicáveis nas etapas do KDD;
		\item Aplicação do processo de KDD utilizando a ferramenta WEKA Knowledge Flow;
		\item Realização dos experimentos numa base de dados pública;
		\item Análise dos resultados.
	\end{enumerate}
\end{frame}

\subsection{Ferramentas}
\begin{frame}
	WEKA (Waikato Enviroment for Knowledge Analysis):
	\begin{itemize}
		\item Ferramenta de auxílio desenvolvida na Nova Zelândia;
		\item WEKA Knowledge Flow permite a vizualização de processos como data-flow.
	\end{itemize}
\end{frame}

\subsection{Base de dados}
\begin{frame}
	\begin{itemize}
		\item Inicialmente descrita por Moro 2016 \cite{moro2016};
		\item A base consiste de postagens publicadas durante o ano de 2014 na página do Facebook de uma marca de cosméticos.
	\end{itemize}

placeholder tabela de dados
\end{frame}