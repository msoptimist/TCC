\section{Introdução}

\subsection{Contexto}
\begin{frame}
	\begin{itemize}
		\item Aumento da presença de serviços digitais na vida cotidiana;
		\item Acumular dados não é o suficiente;
		\item Fayyad 1996 \cite{fayyad1996} diz que neste cenário que há uma grande necessidade de novas teorias e ferramentas computacionais a fim de auxiliar na extração de conhecimento dos grandes e cada vez maiores volumes de dados.
	\end{itemize}
\end{frame}

\subsection{Processo de KDD}

\begin{frame}
	\begin{itemize}
		\item Knowledge Discovery in Databases - cunhado por Shapiro em 1989.
		\item Processo iterativo e não trivial dependente de interações do usuário.
		\item Tem como objetivo identificar informações válidas, novas, potencialmente úteis e compreensíveis em um grupo de dados;
		\item Mineração de dados é uma das etapas que mais recebe destaque.
	\end{itemize}

placeholder de imagem de kdd
\end{frame}