% CAPA---------------------------------------------------------------------------------------------------

% ORIENTAÇÕES GERAIS-------------------------------------------------------------------------------------
% Caso algum dos campos não se aplique ao seu trabalho, como por exemplo,
% se não houve coorientador, apenas deixe vazio.
% Exemplos: 
% \coorientador{}
% \departamento{}

% DADOS DO TRABALHO--------------------------------------------------------------------------------------
\titulo{Título do Trabalho: Subtítulo do Trabalho}
\autor{Nome Completo do Autor}
\autorcitacao{SOBRENOME, Nome} % Sobrenome em maiúsculo
\local{Guarapuava}
\data{2016}

% NATUREZA DO TRABALHO-----------------------------------------------------------------------------------
\projeto{Proposta de Trabalho de Conclusão de Curso}

% TÍTULO ACADÊMICO---------------------------------------------------------------------------------------
% - Bacharel ou Tecnólogo
\tituloAcademico{Tecnólogo}

% DADOS DA INSTITUIÇÃO-----------------------------------------------------------------------------------
% Coloque o nome do curso de graduação em "programa"
% Formato para o logo da Instituição: \logoinstituicao{<escala>}{<caminho/nome do arquivo>}
\instituicao{Universidade Tecnológica Federal do Paraná}
\departamento{Coint - Tecnologia em Sistemas para Internet}
\programa{Curso de Tecnologia em Sistemas para Internet}
\logoinstituicao{0.2}{dados/figuras/logo-instituicao.png} 

% DADOS DOS ORIENTADORES---------------------------------------------------------------------------------
\orientador{Nome do orientador}
%\orientador[Orientadora:]{Nome da orientadora}
\instOrientador{}

\coorientador{Nome do coorientador}
%\coorientador[Coorientadora:]{Nome da coorientadora}
\instCoorientador{}
