% REVISÃO DE LITERATURA--------------------------------------------------------

\chapter{REVISÃO SISTEMATICA DA LITERATURA}
\label{chap:revisaoSistematica}
Neste Capítulo encontra-se um levantamento bibliográfico a fim de obter um panorama abrangente dos estudos a cerca de mineração de dados em bases retiradas de redes sociais e os métodos e técnicas que são utilizados sobre esses dados. Na Seção \ref{sec:metodoRevisao} está descrito o método de revisão sistemáticas utilizado neste trabalho. A Seção \ref{sec:aplicacaoRevisao} contém a aplicação do método e os resultados obtidos do mesmo. Por fim, a Seção \ref{sec:consideracoesFinaisRevisao} apresenta as considerações finais a cerca do Capítulo.

\section{MÉTODO DE REVISÃO SISTEMÁTICA}
\label{sec:metodoRevisao}
Explicação do método de revisão sistemática. Baseado em PAGANI 2015 dividido em algumas etapas: definição da intenção da pesquisa e características dos tipos de trabalhos; definição palavras chaves e busca preliminar exploratória; busca efetiva e gerenciamento de bibliografia; primeiras filtragens; ordenação dos artigos filtrados; leitura e análise integral.

\section{APLICAÇÃO DO MÉTODO}
\label{sec:aplicacaoRevisao}
Nesta Seção está a aplicação do método utilizada neste trabalho.

\subsection{Estabelecimento da inteção da pesquisa}
\label{subsec:intencaoPesquisaRevisao}

\subsection{Definição de palavras chaves e busca preliminar exploratória}
\label{subsec:buscaExploratoriaRevisao}

Abaixo encontram-se as bases que foram usadas para a pesquisa de artigos da área de descoberta de conhecimento e mineração de dados

\begin{quadro}[!htb]
    \centering
    \caption{Bases de dados pesquisadas.\label{qua:basesArtigos}}
    \begin{tabular}{|p{7cm}|p{7cm}|}
		\hline
		\textbf{Base de dados} & \textbf{Sites}                                           \\ \hline
		\textit{arXiv}              & \url{https://arxiv.org/}              \\ \hline
		\textit{Emerald Insight}        & \url{https://www.emeraldinsight.com/} \\ \hline
		\textit{IEEE Xplore}             & \url{https://ieeexplore.ieee.org}     \\ \hline
		\textit{Science Direct}         & \url{https://www.sciencedirect.com/}  \\ \hline
		\textit{Springer}               & \url{https://link.springer.com/}      \\ \hline
    \end{tabular}
    \fonte{Autoria própria}
\end{quadro}


Foi aplicado um filtro de busca.

arXiv: query ``social media and data mining'' no abstract entre os anos 2000 e 2018 na área de ciência da computação

Emerald Insight: query ``social media and data mining'' no abstract entre os anos 2000 e 2018

IEEE Xplore: query ``social media and data mining'' no abstract entre os anos 2000 e 2019, filtrando por journals \& magazines e conferences

ScienceDirect: query ``social media data mining'' no abstract entre os anos 2000 e 2018, filtrando research e review articles

Springer: query ``social media data mining'' entre os anos 2000 e 2018, filtrados por english, computer science, artificial intelligence, data mining and knowledge discovery, article

\subsection{Busca efetiva e gerenciamento de bibliografia}
\label{subsec:buscaEfetivaRevisao}

Foi utilizado o \textit{software} gerenciador de bibliografia Zotero para facilitar a vizualição e processo de filtragem dos artigos. Após a busca nas bases citadas na Tabela REFERENCIA PARA AS TABELAS E QUADROS foram encontrados o seguinte numero de artigos

\begin{table}[!htb]
    \centering
    \caption[Resultado das buscas por artigos nas bases selecionadas]{Resultado das buscas.
    \label{tab:resultadosBuscas1}}
    \begin{tabular}{lr}
        \toprule
           \textbf{Base de dados} & \textbf{Resultados} \\ \hline
        \midrule
            \textit{arXiv.org}              & 75                   \\
		\textit{Emerald Insight}        & 30                   \\
		\textit{IEEEXplore}             & 457                   \\
		\textit{Science Direct}         & 236                   \\
		\textit{Springer}               & 320                   \\ 
	\midrule
		Total:				& 1118	\\ \hline
        \bottomrule
    \end{tabular}
\end{table}

\subsection{Procedimento de filtragem}
\label{subsec:filtragemRevisao}

Após os processos de deleção de artigos repetidos e seleção dos artigos relevantes ficamos com a quantidade de artigos mostrados an tabela abaixo.

\begin{table}[!htb]
    \centering
    \caption[Resultado dos filtros aplicados nas buscas por artigos]{Resultado dos filtos.
    \label{tab:resultadosFiltros1}}
    \begin{tabular}{lr}
        \toprule
           \textbf{Base de dados} & \textbf{Resultados} \\ \hline
        \midrule
            \textit{arXiv.org}              & 0                  \\
		\textit{Emerald Insight}        & 4                   \\
		\textit{IEEEXplore}             & 8                  \\
		\textit{Science Direct}         & 3                    \\
		\textit{Springer}               & 3                    \\ 
	\midrule
		Total:			& 18			\\ \hline
        \bottomrule
    \end{tabular}
\end{table}

\subsection{Ordenação dos artigos}
\label{subsec:ordenacaoRevisao}

\subsection{Leitura e análise integral dos artigos}
\label{subsec:leituraIntegralRevisao}

\section{CONSIDERAÇÕES FINAIS DO CAPÍTULO}
\label{sec:consideracoesFinaisRevisao}
Considerações finais.