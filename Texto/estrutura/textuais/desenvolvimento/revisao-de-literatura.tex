% REVISÃO DE LITERATURA--------------------------------------------------------

\chapter{REVISÃO SISTEMATICA DA LITERATURA}
\label{chap:revisaoSistematica}
Revisão dos trabalhos relacionados da área

\section{MÉTODO DE REVISÃO SISTEMÁTICA}
\label{sec:metodoRevisao}
Explicação do método de revisão sistemática

\section{APLICAÇÃO DO MÉTODO}
\label{sec:aplicacaoRevisao}
Abaixo encontram-se as bases que foram usadas para a pesquisa de artigos da área de descoberta de conhecimento e mineração de dados

\begin{quadro}[!htb]
    \centering
    \caption{Bases de dados pesquisadas.\label{qua:basesArtigos}}
    \begin{tabular}{|p{7cm}|p{7cm}|}
		\hline
		\textbf{Base de dados} & \textbf{Sites}                                           \\ \hline
		\textit{arXiv.org}              & \textless{}https://arxiv.org/\textgreater{}              \\ \hline
		\textit{Emerald Insight}        & \textless{}https://www.emeraldinsight.com/\textgreater{} \\ \hline
		\textit{IEEEXplore}             & \textless{}https://ieeexplore.ieee.org\textgreater{}     \\ \hline
		\textit{Science Direct}         & \textless{}https://www.sciencedirect.com/\textgreater{}  \\ \hline
		\textit{Springer}               & \textless{}https://link.springer.com/\textgreater{}      \\ \hline
    \end{tabular}
\end{quadro}


Foi aplicado um filtro de busca.

arXiv: query ``social media and data mining'' no abstract entre os anos 2000 e 2018 na área de ciência da computação

Emerald Insight: query ``social media and data mining'' no abstract entre os anos 2000 e 2018

IEEE Xplore: query ``social media and data mining'' no abstract entre os anos 2000 e 2019, filtrando por journals \& magazines e conferences

ScienceDirect: query ``social media data mining'' no abstract entre os anos 2000 e 2018, filtrando research e review articles

Springer: query ``social media data mining'' entre os anos 2000 e 2018, filtrados por english, computer science, artificial intelligence, data mining and knowledge discovery, article

\begin{table}[!htb]
    \centering
    \caption[Resultado das buscas por artigos nas bases selecionadas]{Resultado das buscas.
    \label{tab:resultadosBuscas1}}
    \begin{tabular}{lr}
        \toprule
           \textbf{Base de dados} & \textbf{Resultados} \\ \hline
        \midrule
            \textit{arXiv.org}              & 75                   \\
		\textit{Emerald Insight}        & 30                   \\
		\textit{IEEEXplore}             & 457                   \\
		\textit{Science Direct}         & 236                   \\
		\textit{Springer}               & 320                   \\ 
	\midrule
		Total:				& 1118	\\ \hline
        \bottomrule
    \end{tabular}
\end{table}

Após os processos de deleção de artigos repetidos e seleção dos artigos relevantes ficamos com a quantidade de artigos mostrados an tabela abaixo.

\begin{table}[!htb]
    \centering
    \caption[Resultado dos filtros aplicados nas buscas por artigos]{Resultado dos filtos.
    \label{tab:resultadosFiltros1}}
    \begin{tabular}{rr}
        \toprule
           \textbf{Base de dados} & \textbf{Resultados} \\ \hline
        \midrule
            \textit{arXiv.org}              & novo valor                   \\
		\textit{Emerald Insight}        & novo valor                   \\
		\textit{IEEEXplore}             & novo valor                  \\
		\textit{Science Direct}         & novo valor                    \\
		\textit{Springer}               & novo valor                    \\ \hline
        \bottomrule
    \end{tabular}
\end{table}