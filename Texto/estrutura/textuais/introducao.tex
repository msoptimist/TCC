% INTRODUÇÃO-------------------------------------------------------------------

% REFERÊNCIAS DESTE CAPÍTULO
%
%	PREDICTING SOCIAL MEDIA PERFORMANCE
%	DOSSIER STATISTA
%	https://www.internetworldstats.com/stats.htm

\chapter{INTRODUÇÃO}
\label{chap:introducao}

Desde o seu surgimento, a Internet têm movimentado o mundo de uma forma que seus idealizadores na década de 60 jamais poderiam imaginar. Segundo dados estimados pelo \textit{site} \citeonline{IWS}, 54,4\% dos 7,6 bilhões de pessoas no planeta estão conectados a rede mundial de computadores, somando assim um total de mais de 4,1 bilhões de usuários \textit{online}. Desse total, 2,62 bilhões fazem uso de algum tipo de rede social e a previsão é que esse número aumente para 3,02 bilhões até o fim de 2021, segundo pesquisas da Statista \citeonline{DSTATISTA}.

Dentro desse universo de redes sociais, o \textit{Facebook} possui a maior base de usuários ativos. Em abril de 2017, segundo o relatório do segundo quadrimestre \textit{GLOBAL DIGITAL STATSHOT} \citeonline{GDS}, publicado através da parceita entre a \textit{We Are Social} e o \textit{Hootsuite}, o \textit{Facebook} atingiu a marca de 1,9 bilhão de usuários ativos mensais e no primeiro quadrimestre de 2018 esse número já passava de 2,2 bilhões segunda pesquisas da Statista \citeonline{DSTATISTA}

	% REVER REFERÊNCIAS

Somente esses números são suficientes para colocar o \textit{Facebook} como principal rede social utilizada, em muitro superando o segundo colocado, o \textit{YouTube} com 1,5 bilhão de usuários ativos mensais. 

Além do crescimento do uso da \textit{Internet} como meio de consumo, os avanços das tecnologias móveis e redes sociais estão possibilitando a geração de quantidades de dados cada vez maiores. Um estudo da IBM \textit{Marketing Cloud} \citeonline{IBMmarketing} descreve que são criados, aproximadamente, 2,5 quintilhões de \textit{bytes} de dados todos os dias e que 90\% de todo o montante de dados presente no mundo hoje foi criado a partir de 2016.

\section{DESCRIÇÃO DO PROBLEMA E MOTIVAÇÃO}
\label{sec:descricaomotivacao}

Tecnologias atuais possibilitam o armazenamento e acesso a essa grande quantidade de dados há um custo muito baixo. Porém, o principal problema associado a um mundo centrado em informações é utilizar os dados brutos coletados \citeonline{surveyKDD}.

Neste cenário que nos encontramos, cada vez mais se mostra necessário o uso de ferramentas computacionais para auxiliar na extração de conhecimentos desses volumes de dados, uma vez que somente acumular dados não necessariamente se traduz em informações úteis e aplicáveis \citeonline{dataMiningToKDD}.

\section{OBJETIVOS}
\label{sec:objetivos}
Esta Seção apresenta o obejtivo geral e os objetivos específicos deste trabalho. Na Subseção \ref{subsec:objGerais} encontra-se o objetivo geral e na Subseção \ref{subsec:objEspecificos} encontram-se os objetivos específicos.

\subsection{Objetivos gerais}
\label{subsec:objGerais}
O objetivo geral deste trabalho é aplicar o KDD numa base de dados retirada do \textit{Facebook} com o intuito de extrair informações a respeito da relação entre os metadados das postagens e as métricas de avaliação geradas pelos algoritmos da rede social em questão.

	% REVER PARÁGRAFO E REFORMATAR OBJETIVOS ESPECÍFICOS EM PARÁGRAFO ÚNICO

\subsection{Objetivos específicos}
\label{subsec:objEspecificos}
Como objetivos específicos deste trabalho têm-se:
\begin{itemize}
	\item compreender o funcionamento do processo de KDD;
	\item analisar as etapas do KDD, identificado técnicas que podem ser aplicadas;
	\item compreender o funcionamento das métricas geradas pelo \textit{Facebook}
	\item aplicar o processo de KDD na base de dados de postagens;
	\item realizar experimentos;
	\item analisar os resultados obtidos por meio de comparação estatística com outros trabalhos da área.
\end{itemize}

\section{ORGANIZAÇÃO DO TRABALHO}
\label{sec:organizacaoTrabalho}
Este trabalho encontra-se dividido nos seguintes sete capítulos:

\begin{itemize}
	\item Capítulo \ref{chap:introducao}: Capítulo introdutório de contextualização do trabalho, apresentando em linhas gerais a situação atual, a motivação que levou a idealização deste trabalho e os objetivos a serem alcançados.

	\item Capítulo \ref{chap:descobertaConhecimento}: Este capítulo aborda os conceitos necessários para compreender o processo de KDD. Contém uma descrição das etapas que o constituem como um todo e entra mais a fundo em conceitos importantes da parte de mineração de dados.

	\item Capítulo \ref{chap:redesSociais}: O capítulo apresenta conceitos e definições a respeito de redes sociais, classificando-as e descrevendo sua relevância fora e dentro da área acadêmica. Também contém uma explicação a respeito de termos da rede social Facebook, sendo esta o local onde os dados utilizados neste trabalho foram retirados.

	\item Capítulo \ref{chap:revisaoSistematica}: Neste capítulo encontra-se uma explicação a respeito do método de revisão metodológica, ressaltando os pontos importantes do processo e a aplicação deste em bases de artigos acadêmicos, permitindo a identificação de trabalhos correlatos que serviram de referência para a elaboração deste.

	\item Capítulo \ref{chap:metodologia}: O capítulo descreve a base de dados utilizada para a realização dos experimentos deste trabalho, bem como as técnicas e métodos a serem aplicados na mesma, seguindo o processo definido no capítulo \ref{chap:descobertaConhecimento}.

	\item Capítulo \ref{chap:resultados}: Este capítulo contém a descrição e análise dos resultados obtidos após a aplicação dos métodos e técnicas descritos no capítulo \ref{chap:metodologia} sobre a base de dados deste trabalho.

	\item Capítulo \ref{chap:conclusao}: O capítulo contém possibilidades de continuações deste trabalho, bem como as considerações finais a respeito dos resultados obtidos e do cumprimento dos objetivos propostos.
\end{itemize}