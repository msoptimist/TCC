\begin{quadro}[!htb]
    \centering
    \caption{Etapas do pré-processamento do KDD.\label{qua:etapasPreProcKDD}}
    \begin{tabular}{|p{7cm}|p{7cm}|}
        \hline
        \textbf{Etapas} & \textbf{Descrição} \\
        \hline
        1. Desenvolver e compreender o domínio da aplicação & Este passo diz respeito ao aprendizado de quaisquer conhecimentos anteriores ao domínio da aplicação além dos objetivos do usuário para o novo conhecimento descoberto. É um passo preparatório para o trabalho com a base de dados em sim. \\ \hline
        2. Criar o conjunto de dados alvo & Envolve a seleção dos atributos e instâncias em que serão aplicadas as tarefas de descoberta de conhecimento, geralmente percorrendo a base a fim de selecionar os dados do subconjunto. \\ \hline
        3. Limpeza dos dados & Este passo consiste em na remoção de \textit{outliers}, limpeza de ruído e inputação de valores faltantes. \\ \hline
        4 Redução de dados e projeção & Consiste na aplicação de métodos de transformação a fim de reduzir as dimensões dos dados, continuam o processo somente com os atributos relevantes ou encontrar representação não variantes dos dados a serem tratados\\
        \hline
    \end{tabular}
    \fonte{\cite{DMKDPAPPROACH}}
\end{quadro}